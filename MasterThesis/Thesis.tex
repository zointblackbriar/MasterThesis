%--------------------------------------------------------------------------------------%--------------------------------------------------------------------------------------
%
%  Global settings, dont change it! (excapt additional \usepackage commands)
%  Always use PDFLatex!
%
%--------------------------------------------------------------------------------------%--------------------------------------------------------------------------------------
\documentclass[a4paper, 12pt, oneside, BCOR=1cm,toc=chapterentrywithdots]{scrbook}

\usepackage{graphicx}           % use for pdfLatex
\usepackage{makeidx} % f\"{u}r Benutzung des Befehls \printindex
\usepackage[colorlinks=false]{hyperref}
\usepackage{tocbibind}
\usepackage{blindtext}
\usepackage{subfigure} 
\usepackage{acronym}

\hypersetup{%
bookmarksnumbered=true, hyperindex=true,
%
%Im Acrobat Reader Subtitel 1. Ebene anzeigen
bookmarksopen=true, bookmarksopenlevel=1,
%
pdfborder=0 0 0 % Keine Box um die Links!
}

% --------------------------------------------------------------
% Force Tables and List to be added in Table of Content
% --------------------------------------------------------------

\renewcommand*{\tableofcontents}{%
  	\begingroup
  	\tocsection
  	\tocfile{\contentsname}{toc}
  	\endgroup
}
\renewcommand*{\listoffigures}{%
  	\begingroup
  	\tocsection
  	\tocfile{\listfigurename}{lof}
  	\endgroup
}
\renewcommand{\listoftables}{
	\begingroup
	\tocsection
	\tocfile{\listtablename}{lot}
	\endgroup
}
\begin{document}

%--------------------------------------------------------------------------------------%--------------------------------------------------------------------------------------
%
%  Here starts the userspace !
%
%--------------------------------------------------------------------------------------%--------------------------------------------------------------------------------------

%--------titlepage
\begin{titlepage}

{
    \begin{center}
        \raisebox{-1ex}{\includegraphics[scale=1.5]{TU_Chemnitz_positiv_gruen.pdf}}\\
    \end{center}
    \vspace{0.5cm}
}

\begin{center}

\LARGE{\textbf{Design and Implementation of an OPC-UA Web Service Integrated to a Closed-Domain Question Answering System}}\\
\vspace{1cm}


\Large{\textbf{Master Thesis}}\\ 
\vspace{1cm}
Submitted in Fulfilment of the\\
Requirements for the Academic Degree\\
M.Sc. \\
\vspace{1cm}
Dept. of Computer Science\\
Chair of Computer Engineering

\vspace{0.5cm}
Orcun Oruc \\
Chemnitz, 18th of February 2019 \\

\end{center}
\vspace{18cm}
Submitted by: Orcun Oruc\\
Student ID: 321448\\
Date: 18.02.2019\\

\vspace{0.1cm}
Design and Implementation of an OPC-UA Web Service Integrated to a Closed-Domain Question Answering System \\
\vspace{1cm}\\
Main Examiner : Prof. Dr. None \\
First Supervising Tutor: Dr. Frank Seifert \\
Second Supervising Tutor 

\end{titlepage}

%---------------------------------------------------------
% Acknowledgements
%---------------------------------------------------------

\addchap*{Acknowledgments}
\text
This research was supported by the Fraunhofer IWU Institute. I specially thank to my colleague Msc. Adrian Singer. Without his guidance and persistent help this thesis would not have been possible. I would like to express the deepest appreciation to my university supervisor Dr. Frank Seifert who provided insight and expertise that greatly assisted the research. I am deeply grateful to my family who supported whatever it costs through all my life. 

I thank Ken Wenzel Msc. for assistance with LinkedFactory and Enilink Systems and for his comments that remarkably improved the manuscript.  
\vspace{1cm}
\\\\
\textbf{Keywords: Thanks, Family, Friends, Society} % Thank you for all

%---------------------------------------------------------
% Abstract
%---------------------------------------------------------

\addchap*{Abstract}
%\blindtext
\text
Impediment: Herr Singer should send his abstract for the 
company's point of view. \\
Semantic Web provides tremendous solutions concerning machine to human interaction. A vast amount of structured linked data and non-structured data is stored in World Wide Web and many companies. Nowadays, interlinked data sources prevailed in every area of World Wide Web and Industrial Internet of Things devices leverage semantic data sources to communicate with high application protocols. State-of-the-art OPC-UA Protocol would lead to use companies conduct survey how to implement efficiently. We propose a semantic web integration service to query questions against closed-domain of ENILINK of Fraunhofer IWU. By means of the queries, the LinkedFactory, which named by Fraunhofer IWU, we elaborate specification of the factory and present to any web users or customers. We will explore 

\\\\
\textbf{Keywords: Keyword1, Keyword2, Keyword3, ...max 5}

%---------------------------------------------------------
% Table of Contents, List of figures, List of Tables
%---------------------------------------------------------

\tableofcontents
\listoffigures
\listoftables

%---------------------------------------------------------
% List of Abbreviations
%---------------------------------------------------------

\twocolumn
\addchap{List of Abbreviations}
\begin{acronym}[Bash]
 \acro{KDE}{K Desktop Environment}
 \acro{SQL}{Structured Query Language}
 \acro{Bash}{Bourne-again shell}
 \acro{JDK}{Java Development Kit}
 \acro{VM}{Virtuelle Maschine}
 \acro{I2C}[I²C]{Inter-Integrated Circuit}
 \acro{KDE}{K Desktop Environment}
 \acro{SQL}{Structured Query Language}
 \acro{Bash}{Bourne-again shell}
 \acro{JDK}{Java Development Kit}
 \acro{VM}{Virtuelle Maschine}
 \acro{I2C}[I²C]{Inter-Integrated Circuit}
 \acro{KDE}{K Desktop Environment}
 \acro{SQL}{Structured Query Language}
 \acro{Bash}{Bourne-again shell}
 \acro{JDK}{Java Development Kit}
 \acro{VM}{Virtuelle Maschine}
 \acro{I2C}[I²C]{Inter-Integrated Circuit}
 \acro{KDE}{K Desktop Environment}
 \acro{SQL}{Structured Query Language}
 \acro{Bash}{Bourne-again shell}
 \acro{JDK}{Java Development Kit}
 \acro{VM}{Virtuelle Maschine}
 \acro{I2C}[I²C]{Inter-Integrated Circuit}
 \acro{KDE}{K Desktop Environment}
 \acro{SQL}{Structured Query Language}
 \acro{Bash}{Bourne-again shell}
 \acro{JDK}{Java Development Kit}
 \acro{VM}{Virtuelle Maschine}
 \acro{I2C}[I²C]{Inter-Integrated Circuit}
 \acro{KDE}{K Desktop Environment}
 \acro{SQL}{Structured Query Language}
 \acro{Bash}{Bourne-again shell}
 \acro{JDK}{Java Development Kit}
 \acro{VM}{Virtuelle Maschine}
 \acro{I2C}[I²C]{Inter-Integrated Circuit}
\end{acronym}

\onecolumn
%---------------------------------------------------------
% Here starts the real work
%---------------------------------------------------------

\chapter{Introduction}
% just insert your text here

%\blindtext[0]

%\blindtext[4]

%\blindtext[1]

When we talk about OPC UA. We can   % Load Data from File intro.tex

\chapter{Tables}
\input{src/example_tables} % Load Data from File example_tables

\chapter{Figures}
\input{src/example_figures} % Load Data from File example_figures

\chapter{Referencing}
% Alternativ just write your text under \chapter like this example

\blindtext \cite{autorenrichtlinien}

\blindtext \footnote{Here is an area for your Notes}

\blindtext \footnote{\cite{lnilatex} Seite 11}
\blindtext \cite{lnilatex}
\blindtext \cite{autorenrichtlinien,pepper1992grundlagen,chen2001audiovisual}


\chapter{Subchapter}

\section{sub 1}
\blindtext[3]
\section{sub 2}
\blindtext[3]
\subsection{sub 2.1}
\blindtext[3]

\subsection{sub 2.2}
\blindtext[3]

%---------------------------------------------------------
% bibliography based on Springer Design
%---------------------------------------------------------

\bibliographystyle{splncs03}
\bibliography{bibliography}

\printindex

\end{document}
